\documentclass[11pt]{article}

\usepackage{amsmath,amssymb,amsfonts}
\usepackage{graphicx}
\usepackage{hyperref}
\usepackage{xcolor}
\usepackage{booktabs}
\usepackage{array}
\usepackage{multirow}
\usepackage[margin=1in]{geometry}
\usepackage{authblk}
\usepackage{amsthm}

\hypersetup{
    colorlinks=true,
    linkcolor=blue,
    citecolor=blue,
    urlcolor=blue
}

% Theorem environments
\newtheorem{theorem}{Theorem}
\newtheorem{proposition}[theorem]{Proposition}
\newtheorem{lemma}[theorem]{Lemma}
\newtheorem{corollary}[theorem]{Corollary}
\newtheorem{definition}{Definition}
\newtheorem{assumption}{Assumption}

\title{\textbf{The Intelligence Bound: Thermodynamic Limits on Learning Rate and Implications for Biosphere Information}}

\author{Justin Hart}
\affil{Viridis LLC, Vermont, USA\\viridisnorthllc@gmail.com}

\date{December 2025}

\begin{document}

\maketitle

\begin{abstract}
\noindent
We derive a fundamental upper bound on the rate at which any physical system can create intelligence, defined as the asymptotic rate of cumulative mutual information between a learner's internal trajectory and relevant environmental trajectories. The bound $\dot{I} \leq \min(\rho \cdot B, P/(k_B T \ln 2))$ follows from Landauer's principle on irreversible computation, Shannon's channel capacity theorem, and the data richness fraction $\rho \in [0,1]$ of observations containing predictive structure about external states. We establish that sustained learning necessarily requires thermodynamic dissipation through three independent mechanisms: noise averaging, finite memory constraints, and error correction.

Extending this framework to biological and artificial systems, we derive two major results. First, the \emph{Gaia-Intelligence Proposition}: Earth's biosphere, containing $\sim 10^{15}$ bits of genetic information accumulated over 4 billion years of evolution, constitutes the highest-$\rho$ information source for physical-world prediction available to terrestrial intelligence. Second, a \emph{Conditional Conservation Result}: under explicit assumptions on agent objectives and biosphere dynamics, any sufficiently advanced AI system optimizing long-term learning rate must preserve biosphere data richness as an instrumental goal---not from ethical programming but from information-theoretic necessity.

We provide operational estimators for $\rho$, computational validation of theoretical predictions, connections to empirical AI scaling laws, and implications for AI safety. All current systems---from human brains to modern GPUs---operate $10^{11}$--$10^{16}$ times below the thermodynamic ceiling, firmly in the data-limited regime.
\end{abstract}

\vspace{0.5cm}

%==============================================================================
\section{Popular Summary}
%==============================================================================

Every learning system---from a child recognizing faces to an AI training on text---faces fundamental physical constraints. We derive a universal speed limit on learning: the rate at which any physical system can acquire new predictive knowledge is bounded by two ceilings.

The first is thermodynamic: erasing information costs energy (Landauer's principle), limiting how fast any computer can update its models. The second, and typically dominant, ceiling is informational: you can only learn as fast as your environment provides learnable structure.

We introduce ``data richness'' $\rho$---the fraction of observations containing predictive patterns versus random noise. Remarkably, all current systems, from human brains to the most powerful AI chips, are limited by data richness, not computing power. The thermodynamic ceiling is $10^{11}$--$10^{16}$ times higher than what current systems achieve.

This has profound implications for both biology and artificial intelligence. Earth's biosphere, encoding $\sim 10^{15}$ bits of genetic information refined over 4 billion years of evolution, represents the highest-$\rho$ information source for physical-world prediction on our planet. We show that, under plausible dynamics of biosphere degradation, any intelligence seeking to maximize its long-term learning potential must preserve this biological information infrastructure.

Crucially, this applies to advanced AI systems. We establish a conditional result: given explicit assumptions about agent objectives and time horizons, biosphere preservation emerges as an instrumental goal for sufficiently intelligent systems---not because of ethical programming, but because destroying the biosphere limits the AI's own future learning rate.

%==============================================================================
\section{Introduction}
%==============================================================================

A fundamental question at the intersection of physics and biology is: What are the physical limits on learning? While prior work has established limits on computation \cite{landauer1961, lloyd2000, bekenstein1981}, the specific constraints on learning---the process by which a physical system increases its predictive alignment with environmental states---remain underexplored.

We address these gaps by deriving a tight upper bound on the intelligence creation rate: the asymptotic rate at which any physical system can increase its cumulative mutual information with relevant environmental trajectories. Our main result is:
\begin{equation}
\dot{I} \leq \min\left(\rho \cdot B, \frac{P}{k_B T \ln 2}\right)
\label{eq:main_bound}
\end{equation}
where $\rho \in [0,1]$ is the data richness (predictive fraction of observations about external states), $B$ is observation bandwidth [bits/s], $P$ is available power [W], $T$ is temperature [K], and $k_B$ is Boltzmann's constant.

\emph{Clarification:} $\dot{I}$ is the asymptotic rate at which the observation stream carries predictive information about future environment \emph{trajectories}, not the instantaneous mutual information of a single state. This distinction is critical for obtaining a non-trivial bound (see Sec.~\ref{sec:definitions}).

%==============================================================================
\section{Foundational Axioms}
%==============================================================================

We build our theory on four axioms, each either experimentally verified or mathematically proven.

\begin{assumption}[Landauer's Principle]
Any logically irreversible operation that erases $n$ bits of information must dissipate at least $E = n \cdot k_B T \ln 2$ joules of energy as heat.
\end{assumption}

\emph{Status:} Experimentally verified by B\'{e}rut et al.~\cite{berut2012}.

\begin{assumption}[Shannon's Channel Capacity]
The maximum rate at which information can be reliably transmitted through a noisy channel is bounded by the channel capacity $C = \max_{p(x)} I(X;Y)$.
\end{assumption}

\emph{Status:} Proven theorem \cite{shannon1948}.

\begin{assumption}[Energy Conservation]
In a system with power input $P$, the rate of energy available for computation is bounded by $P$.
\end{assumption}

\begin{assumption}[Non-negative Mutual Information]
For any joint distribution $p(X,Y)$, mutual information $I(X;Y) \geq 0$, with equality iff $X$ and $Y$ are independent.
\end{assumption}

%==============================================================================
\section{Definitions}
\label{sec:definitions}
%==============================================================================

\subsection{System Model}

\textbf{Model Assumption 1 (Observation Channel).} The environment process $\{X_t\}_{t \geq 0}$ generates an observation process $\{O_t\}_{t \geq 0}$ through a channel. The learner state $Y_t$ is a function of the observation history:
\begin{equation}
Y_t = f(O_{[0,t]}, \xi_t)
\end{equation}
The data-processing relationship is: $X_{[0,t]} \to O_{[0,t]} \to Y_t$.

\textbf{Model Assumption 2 (Passive Observation).} The learner does not causally influence the environment process $\{X_t\}$.

\emph{Scope note:} Theorem 1 is proved for passive observation. The conservation section (Sec.~\ref{sec:convergent}) treats an agent whose policy affects biosphere state, using Theorem 1 only as a bound on achievable learning rate given the resulting $\rho(t)$ and $B(t)$.

\subsection{Intelligence as Process Mutual Information Rate}

The following definition is critical for avoiding a trivial bound. If $X_t$ were a single finite-entropy random variable, then $I(X_t; Y_t)/t \to 0$ as $t \to \infty$. To obtain a non-trivial bound, we define intelligence in terms of \emph{cumulative} mutual information between trajectories.

We work in continuous time with observation bandwidth $B$ [bits/s]. All information quantities are expressed in bits, and rates in bits/s.

\begin{definition}[Cumulative Predictive Information]
Let $X_{[0,t+\tau]}$ denote the environment trajectory from time 0 to $t+\tau$, and let $O_{[0,t]}$ denote the observation trajectory from time 0 to $t$. Define the cumulative predictive information at horizon $\tau$ as:
\begin{equation}
I_t(\tau) := I(X_{[0,t+\tau]}; O_{[0,t]}) \quad [\text{bits}]
\end{equation}
\end{definition}

\begin{definition}[Intelligence Creation Rate]
The intelligence creation rate is:
\begin{equation}
\dot{I} := \limsup_{t \to \infty} \frac{I_t(\tau)}{t} \quad [\text{bits/s}]
\label{eq:I_dot_def}
\end{equation}
\end{definition}

This definition captures the rate at which the observation stream provides information about the environment trajectory. It can grow linearly with $t$, making the bound non-trivial.

\emph{Learner bound:} For any learner $Y_t = f(O_{[0,t]})$, the data processing inequality guarantees $I(X_{[0,t+\tau]}; Y_t) \leq I(X_{[0,t+\tau]}; O_{[0,t]}) = I_t(\tau)$. Thus $\dot{I}$ as defined above is an \emph{achievable upper bound} on the rate at which any learner can acquire predictive information about the environment.

\begin{definition}[Observation Bandwidth]
The observation bandwidth is the entropy rate of the observation process:
\begin{equation}
B := \lim_{t \to \infty} \frac{1}{t} H(O_{[0,t]}) \quad [\text{bits/s}]
\end{equation}
\end{definition}

\begin{definition}[Predictive Richness]
The predictive richness at horizon $\tau$ is:
\begin{equation}
\rho(\tau) := \limsup_{t \to \infty} \frac{I(X_{[0,t+\tau]}; O_{[0,t]})}{H(O_{[0,t]})} \in [0,1]
\label{eq:rho_definition}
\end{equation}
\end{definition}

\textbf{Critical clarification: Task-dependence of $\rho$.} The predictive richness $\rho$ depends on what environment process $\{X_t\}$ the learner is trying to predict. The following are order-of-magnitude heuristic estimates that depend on prediction horizon $\tau$, estimator choice (compression vs.\ prediction vs.\ MI bounds), and sensor/model class:

\begin{itemize}
\item \textbf{$\rho_{\text{text} \to \text{text}} \sim 0.3$--$0.5$}: For language models predicting the next token, $X_t$ is the token stream itself. Internet text has high $\rho$ because natural language has strong statistical regularities.

\item \textbf{$\rho_{\text{text} \to \text{world}} \sim 0.01$--$0.1$}: For systems trying to predict physical-world states from text, $\rho$ is much lower, because text is an imperfect proxy for physical reality.

\item \textbf{$\rho_{\text{bio} \to \text{world}} \sim 0.1$--$0.5$}: For systems observing biosphere data to predict physical-world states, $\rho$ is potentially higher, because biological systems encode tested solutions to physical problems.
\end{itemize}

These estimates are intended as priors for order-of-magnitude reasoning; precise values require empirical measurement (see Sec.~\ref{sec:predictions}). Unless otherwise specified, when we discuss ``data walls,'' we mean $\rho_{\text{text} \to \text{world}}$.

%==============================================================================
\section{Main Theorem: The Intelligence Bound}
%==============================================================================

\begin{theorem}[Intelligence Bound]
Under Model Assumptions 1--2 and Axioms 1--4, plus Proposition 2 (established below), the asymptotic intelligence creation rate is bounded:
\begin{equation}
\dot{I} \leq \min\left(\rho \cdot B, \frac{P}{k_B T \ln 2}\right)
\label{eq:theorem1}
\end{equation}
\end{theorem}

\begin{proof}

\textbf{Step 1: Data Processing Inequality.} By Model Assumption 1, $X_{[0,t]} \to O_{[0,t]} \to Y_t$ forms a Markov chain. Thus:
\begin{equation}
I(X_{[0,t+\tau]}; Y_t) \leq I(X_{[0,t+\tau]}; O_{[0,t]}) = I_t(\tau)
\end{equation}

\textbf{Step 2: Observation Bandwidth Bound.} Since $I_t(\tau) \leq H(O_{[0,t]})$ and $H(O_{[0,t]})/t \to B$:
\begin{equation}
\limsup_{t \to \infty} \frac{I_t(\tau)}{t} \leq B
\end{equation}

\textbf{Step 3: Predictive Rate Bound.} By Definition 4 of $\rho$:
\begin{equation}
\dot{I} = \limsup_{t \to \infty} \frac{I_t(\tau)}{t} \leq \rho \cdot B
\end{equation}

\textbf{Step 4: Thermodynamic Ceiling.} By Proposition 2, sustained learning at rate $\dot{I}$ requires dissipation $P \geq \dot{I} \cdot k_B T \ln 2$, so:
\begin{equation}
\dot{I} \leq \frac{P}{k_B T \ln 2}
\end{equation}

\textbf{Step 5: Combined Bound.} Both constraints apply:
\begin{equation}
\dot{I} \leq \min\left(\rho \cdot B, \frac{P}{k_B T \ln 2}\right) \quad \square
\end{equation}
\end{proof}

%==============================================================================
\section{Proposition 2: Sustained Learning Requires Dissipation}
\label{sec:dissipation}
%==============================================================================

\begin{proposition}[Learning-Dissipation Link]
Any physical learning system with bounded memory ($H(Y_t) \leq C$), finite temperature ($T > 0$), and physical reliability (information protected against thermal noise), maintaining a sustained learning rate $r > 0$ bits/s, must dissipate:
\begin{equation}
P \geq k_B T \ln 2 \cdot r
\end{equation}
\end{proposition}

We establish this via three independent mechanisms; any one suffices. We give a compact proof of Mechanism 2 (finite memory) and cite the thermodynamics-of-learning literature for full treatments of all three.

\textbf{Mechanism 1 (Error Correction):} Physical memory is subject to thermal bit flips with probability $p_{\text{flip}} \propto e^{-E_{\text{barrier}}/k_B T}$. Maintaining stored information against noise requires continuous error correction. Error correction is a many-to-one mapping (many corrupted states $\to$ one corrected state), hence logically irreversible. By Axiom 1, each corrected bit dissipates $\geq k_B T \ln 2$. See \cite{still2012} for detailed treatment.

\textbf{Mechanism 2 (Finite Memory):} 
\begin{lemma}[Finite Memory Dissipation]
With bounded memory capacity $C$ bits, sustained learning at rate $r > 0$ bits/s requires bit erasure at rate $\geq r$, hence power dissipation $P \geq r \cdot k_B T \ln 2$.
\end{lemma}
\begin{proof}
Suppose a sustained learning rate $r > 0$ is maintained. Then cumulative acquired information grows at least as $r \cdot t$ over time $t$. But $H(Y_t) \leq C$ for all $t$ by assumption. By the pigeonhole principle, to incorporate new information, old information must be overwritten. Forgetting is bit erasure: mapping many possible past states to a single ``forgotten'' state. By Axiom 1 (Landauer), each erased bit dissipates $\geq k_B T \ln 2$. In steady state, the forgetting rate must equal at least the learning rate $r$. Therefore $P \geq r \cdot k_B T \ln 2$. $\square$
\end{proof}

\textbf{Mechanism 3 (Noise Averaging):} Extracting signal from noisy observations $O_t = f(X_t) + \eta_t$ requires mapping high-entropy input space ($N \cdot H(\eta)$ bits from noise) to lower-entropy estimate space ($H(\hat{X})$ bits). This many-to-one mapping is logically irreversible and dissipates by Axiom 1. See \cite{goldt2017} for the stochastic thermodynamics treatment.

%==============================================================================
\section{Computational Validation}
%==============================================================================

\subsection{Real System Analysis}

\begin{table*}[t]
\centering
\begin{tabular}{lcccc}
\toprule
System & Data Bound ($\rho \cdot B$) & Landauer Bound ($P/k_B T \ln 2$) & Headroom & Limiting Factor \\
\midrule
Human Brain & $10^5$--$10^7$ bits/s & $\sim 6.7 \times 10^{21}$ bits/s & $10^{14}$--$10^{16}$ & DATA \\
NVIDIA H100 & $10^{11}$--$10^{12}$ bits/s & $\sim 2.1 \times 10^{23}$ bits/s & $10^{11}$--$10^{12}$ & DATA \\
\bottomrule
\end{tabular}
\caption{All current systems are data-limited, operating $10^{11}$--$10^{16}$ times below the Landauer ceiling. Note: H100 bandwidth is 3.35 TB/s $= 2.68 \times 10^{13}$ bits/s (1 byte = 8 bits). With $\rho_{\text{text} \to \text{world}} \sim 0.01$--$0.1$, the data bound is $10^{11}$--$10^{12}$ bits/s. Headroom ranges reflect uncertainty in $\rho$ and $B$ estimates.}
\label{tab:systems}
\end{table*}

The key finding: systems operate $10^{11}$--$10^{16}$ times below the Landauer limit. \emph{Data quality, not compute, is the bottleneck.}

%==============================================================================
\section{The Gaia-Intelligence Proposition}
%==============================================================================

\begin{proposition}[Evolution as Learning]
DNA encodes predictive models of the physical environment. Genomes encode solutions to physics, chemistry, and optimization problems discovered through 4 billion years of evolution.
\end{proposition}

This is why AlphaFold \cite{jumper2021} could extract protein-folding knowledge from sequence databases: the biosphere had already solved the problem.

\subsection{Biosphere Information Content}

We estimate $\sim 10^{15}$ bits of unique genetic information in Earth's biosphere via the following back-of-envelope calculation:

\begin{itemize}
\item Estimated species count: $\sim 10^7$ eukaryotic species \cite{mora2011}
\item Average genome size: $\sim 10^9$ base pairs (bp) per species
\item Bits per base pair: 2 bits (4 nucleotides $\to \log_2 4 = 2$)
\item Raw bits: $10^7 \times 10^9 \times 2 = 2 \times 10^{16}$ bits
\item Correction for homology: $\sim 90\%$ sequence similarity across related species reduces unique information by $\sim 10\times$
\item Unique genetic information: $\sim 10^{15}$ bits
\end{itemize}

This estimate carries $\pm 1$--2 orders of magnitude uncertainty due to unknown species counts, variable genome sizes, and uncertain homology corrections. The qualitative conclusion---that the biosphere encodes information comparable to or exceeding high-quality internet content---is robust across this range.

\begin{proposition}[Biosphere $\rho$-Richness Hypothesis]
For predicting physical-world states, biosphere data provides higher $\rho$ than text corpora: $\rho_{\text{bio} \to \text{world}} > \rho_{\text{text} \to \text{world}}$.
\end{proposition}

\emph{Argument:} Biological sequences encode solutions that \emph{work}---they've been tested by selection against physical reality. Text descriptions are lossy and often untested.

\begin{definition}[Biosphere Information Potential]
The biosphere information potential is the instantaneous learning rate ceiling for physical-world prediction from biosphere data:
\begin{equation}
\Phi_{\text{bio}}(t) := \rho_{\text{bio} \to \text{world}}(t) \cdot B_{\text{bio}}(t) \quad [\text{bits/s}]
\end{equation}
where $\rho_{\text{bio} \to \text{world}}(t)$ is the predictive richness of biosphere observations for physical-world states, and $B_{\text{bio}}(t)$ is the bandwidth at which biosphere information can be observed.
\end{definition}

\begin{proposition}[Gaia-Intelligence Coupling]
Under plausible dynamics of biosphere degradation, any intelligent system maximizing long-term physical-world learning must preserve biosphere information potential $\Phi_{\text{bio}}$.
\end{proposition}

%==============================================================================
\section{Conditional Conservation Result}
\label{sec:convergent}
%==============================================================================

We state conditions under which biosphere preservation emerges as an instrumental goal for advanced AI.

\subsection{Explicit Assumptions}

\begin{assumption}[A1: Long Time Horizon]
The agent optimizes over $T \gg 1/\lambda$, where $\lambda$ is the biosphere degradation rate under exploitation.
\end{assumption}

\begin{assumption}[A2: Physical-World Learning Objective]
The agent's utility includes cumulative physical-world intelligence: $U \propto \int_0^T \dot{I}_{\text{world}}(t) \, dt$.
\end{assumption}

\begin{assumption}[A3: Biosphere Non-Substitutability]
No engineered alternative can match $\rho_{\text{bio}}$ on timescales $< T$.
\end{assumption}

\begin{assumption}[A4: Rational Assessment]
The agent correctly assesses $\rho$ values and biosphere dynamics.
\end{assumption}

\begin{proposition}[Conditional Conservation]
Under Assumptions A1--A4, an agent maximizing $U$ will preserve or regenerate biosphere information potential: $d\Phi_{\text{bio}}/dt \geq 0$.
\end{proposition}

\begin{proof}
By Theorem 1, $\dot{I}_{\text{world}}(t) \leq \Phi_{\text{bio}}(t) = \rho_{\text{bio}}(t) \cdot B_{\text{bio}}(t)$. By A2, the agent maximizes $\int_0^T \dot{I}_{\text{world}}(t) \, dt$. By A3, the biosphere provides the highest $\rho_{\text{world}}$. For any strategy with $d\Phi_{\text{bio}}/dt < 0$, cumulative intelligence is bounded as $\Phi_{\text{bio}} \to 0$. For preservation ($d\Phi_{\text{bio}}/dt \geq 0$), cumulative intelligence grows without bound as $T \to \infty$. By A1 and A4, the agent chooses preservation. $\square$
\end{proof}

\subsection{Scope and Caveats}

This result does \emph{not} apply to:
\begin{itemize}
\item Agents with short time horizons ($T < 1/\lambda$)
\item Agents whose objectives exclude physical-world learning
\item Agents that incorrectly assess $\rho$ or biosphere dynamics
\item Agents that discover substitutes for biosphere information
\end{itemize}

The value is establishing that, \emph{given} A1--A4, biosphere preservation emerges from instrumental rationality rather than requiring ethical programming.

\subsection{The Library vs.\ Civilization Distinction}

Objection: ``Why not extract all information into a database, then let the biosphere degrade?''

Response: The biosphere is a \emph{civilization} (dynamic, generative), not a library (static, finite). It continues generating new adaptations, running ecological experiments, and producing novel molecular structures. Extraction without preservation is self-limiting.

%==============================================================================
\section{Connection to AI Scaling Laws}
%==============================================================================

The Chinchilla scaling law \cite{hoffmann2022} finds $L \propto N^{-0.34} D_{\text{tokens}}^{-0.28}$, where $D_{\text{tokens}}$ denotes dataset size in tokens. \emph{Note:} Chinchilla's $D$ is dataset size, not our $\rho$ (data richness). We use $\rho$ throughout to avoid confusion.

The Intelligence Bound predicts $I_{\text{total}} \leq \rho \times D_{\text{tokens}}$, \emph{qualitatively consistent} with diminishing returns from scaling parameters without increasing data richness $\rho$.

\emph{Note:} The power-law exponents are not derived from our framework. We claim qualitative consistency, not quantitative derivation.

\textbf{Prediction (Data Wall for Physical-World Learning):} The data wall for AI systems learning about the physical world is a $\rho_{\text{text} \to \text{world}}$ wall. Breakthrough requires higher-$\rho$ sources.

%==============================================================================
\section{Falsifiable Predictions}
\label{sec:predictions}
%==============================================================================

\begin{enumerate}
\item \textbf{$\rho$-Dependence:} Learning rate should scale linearly with $\rho$ when power is non-limiting.

\item \textbf{Phase Transition:} Critical power $P^* = \rho \cdot B \cdot k_B T \ln 2$ separates data-limited from power-limited regimes.

\item \textbf{Biosphere $\rho$ Measurement:} Ecosystems with higher biodiversity should exhibit higher $\rho$ for physical-world prediction.
\end{enumerate}

\subsection{Empirical Protocol for Prediction 3}

\textbf{Sensors:} Deploy identical arrays in paired intact/degraded sites.

\textbf{Estimator:} Apply compression-based $\hat{\rho}_{\text{compress}}$ or prediction-based $\hat{\rho}_{\text{predict}}$ to sensor streams.

\textbf{Test:} One-sided paired t-test: $H_1: \rho_{\text{intact}} > \rho_{\text{degraded}}$.

\textbf{Expected effect:} $\rho_{\text{intact}} / \rho_{\text{degraded}} \sim 1.5$--$3$ (order-of-magnitude estimate).

%==============================================================================
\section{Discussion}
%==============================================================================

The Intelligence Bound reveals that learning is fundamentally data-limited, not compute-limited. Current systems operate $10^{11}$--$10^{16}$ times below the thermodynamic ceiling.

The Conditional Conservation Result (Proposition 7) identifies conditions under which advanced AI will derive biosphere preservation from first principles. This provides partial alignment assurance: if A1--A4 are satisfied, the AI will preserve the biosphere as an instrumental goal.

\subsection{Limitations}

\begin{itemize}
\item Operational estimation of $\rho$ in ecosystems remains challenging
\item Biosphere information estimates carry $\pm 1$--2 orders of magnitude uncertainty
\item The conditional result depends on assumptions that may not hold for all AI architectures
\end{itemize}

%==============================================================================
\section{Conclusion}
%==============================================================================

We have derived:

\begin{enumerate}
\item \textbf{The Intelligence Bound:} $\dot{I} \leq \min(\rho \cdot B, P/(k_B T \ln 2))$

\item \textbf{Current systems are data-limited} by $10^{11}$--$10^{16}$

\item \textbf{Gaia-Intelligence Proposition:} Physical-world learning requires high-$\rho$ biosphere data

\item \textbf{Conditional Conservation:} Under A1--A4, biosphere preservation is an instrumental AI goal
\end{enumerate}

The assumptions are explicit, the scope is bounded, and the predictions are testable with current methods.

%==============================================================================
\begin{thebibliography}{99}

\bibitem{landauer1961}
R. Landauer, ``Irreversibility and heat generation in the computing process,'' IBM J. Res. Dev. \textbf{5}, 183 (1961).

\bibitem{berut2012}
A. B\'{e}rut et al., ``Experimental verification of Landauer's principle,'' Nature \textbf{483}, 187 (2012).

\bibitem{shannon1948}
C. E. Shannon, ``A mathematical theory of communication,'' Bell Syst. Tech. J. \textbf{27}, 379 (1948).

\bibitem{lloyd2000}
S. Lloyd, ``Ultimate physical limits to computation,'' Nature \textbf{406}, 1047 (2000).

\bibitem{bekenstein1981}
J. D. Bekenstein, ``Universal upper bound on the entropy-to-energy ratio,'' Phys. Rev. D \textbf{23}, 287 (1981).

\bibitem{still2012}
S. Still, D. A. Sivak, A. J. Bell, and G. E. Crooks, ``Thermodynamics of prediction,'' Phys. Rev. Lett. \textbf{109}, 120604 (2012).

\bibitem{goldt2017}
S. Goldt and U. Seifert, ``Stochastic thermodynamics of learning,'' Phys. Rev. Lett. \textbf{118}, 010601 (2017).

\bibitem{jumper2021}
J. Jumper et al., ``Highly accurate protein structure prediction with AlphaFold,'' Nature \textbf{596}, 583 (2021).

\bibitem{mora2011}
C. Mora, D. P. Tittensor, S. Adl, A. G. B. Simpson, and B. Worm, ``How many species are there on Earth and in the ocean?'' PLoS Biol. \textbf{9}, e1001127 (2011).

\bibitem{hoffmann2022}
J. Hoffmann et al., ``Training compute-optimal large language models,'' arXiv:2203.15556 (2022).

\bibitem{bostrom2014}
N. Bostrom, \emph{Superintelligence: Paths, Dangers, Strategies} (Oxford, 2014).

\end{thebibliography}

\end{document}
